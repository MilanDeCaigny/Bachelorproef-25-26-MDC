    \tableofcontents
    
    \section{Inleiding}%
    \label{sec:inleiding}
    
    Moderne havenbedrijven zijn sterk afhankelijk van een complex IT-landschap dat de operationele processen ondersteunt, zoals goederenstromen, transportplanning en magazijnbeheer. Binnen dit landschap spelen tientallen componenten een sterke rol in het realiseren van efficiënte en tijdige operaties, waaronder WMS~\autocite{Matthias2025}, TMS~\autocite{SAPn.d.}  en infrastructuursystemen. Omdat er in dit tijdperk veel gedigitaliseerd is, heeft elke IT-verstoring een directe impact op de prestaties van een havenbedrijf.
    In de praktijk blijkt dat IT-storingen vaak te laat worden gedetecteerd en daardoor niet altijd tijdig worden opgelost. Een belangrijke oorzaak is dat het monitoren per systeem gebeurt, hierdoor wordt er geen geïntegreerd overzicht gemaakt over het end-to-end proces~\autocite{Shaikh2025}. Volgens een recent onderzoek naar monitoringsstrategieën kan end-to-end monitoring organisaties helpen om incidenten sneller te identificeren, begrijpen en verhelpen~\autocite{Mireles2024}. Dit onderstreept de nood aan een oplossing die verder gaat dan klassieke monitoringtools.
    Voor een dienstverlener in de haven vormt dit probleem meer dan een technische uitdaging: vertraagde detectie van incidenten leidt tot operationele vertragingen, hogere kosten en verminderde betrouwbaarheid richting klanten~\autocite{Shaikh2025}. De impact van IT-storingen op logistieke processen blijkt ook uit praktijkvoorbeelden. Zo leidde de grootschalige NotPetya-aanval op A.P. Møller-Maersk in 2017 tot wereldwijde verstoringen van terminals en containerverwerking, waardoor havenoperaties gedurende meerdere dagen tot weken werden beïnvloed. De totale financiële schade werd geschat tussen 200 en 300 miljoen dollar, wat illustreert hoe sterk logistieke ketens afhankelijk zijn van de beschikbaarheid en stabiliteit van IT-systemen~\autocite{Mathews2017}. Ook recentere incidenten bevestigen deze afhankelijkheid. In juli 2024 veroorzaakte een fout in een software-update van CrowdStrike een wereldwijde IT-outage die Microsoft Windows-systemen trof, met tijdelijke verstoringen bij containerterminals zoals Baltic Hub en de Port of Felixstowe, en operationele hinder bij bedrijven binnen de haven van Rotterdam. Dit toont aan dat zelfs kortdurende IT-storingen directe operationele gevolgen kunnen hebben voor havenactiviteiten~\autocite{Mandra2024}. Het gebrek aan zichtbaarheid en alerting verhindert dat de IT-operationele teams storingen tijdig kunnen aanpakken. Daardoor is er een duidelijke behoefte aan een geïntegreerde monitoringsoplossing die het volledige plaatje in kaart brengt, proactief alarmeert en de stabiliteit verhoogt. 
    Om dit probleem te onderzoeken, worden in dit onderzoek de volgende onderzoeksvragen opgesteld, onderverdeeld  in een probleemdomein en een oplossingsdomein:

    \textbf{Probleemdomein:}
    \begin{enumerate}
        \item Welke IT-systemen en infrastructuurcomponenten zijn essentieel voor de uitvoering van de end-to-end processen binnen een havenbedrijf?
        \item Welke types IT-incidenten hebben de grootste negatieve impact op operationele KPI’s?
        \item Wat zijn de tekortkomingen van de huidige monitoringsmethodes in het bieden van overzicht en samenhang binnen end-to-end processen?
    \end{enumerate}

    \textbf{Oplossingsdomein:}
    \begin{enumerate}\setcounter{enumi}{3}
        \item Hoe kan een end-to-end monitoringsoplossing worden ontworpen zodat metrics, logs en traces efficiënt geïntegreerd worden binnen één centraal monitoringplatform?
        \item Hoe kan een end-to-end monitoringsoplossing worden ingericht om IT-incidenten tijdig te detecteren en relevante alerts te genereren voor IT-operationele teams?
        \item Hoe kan een end-to-end monitoringsoplossing zichtbaar maken hoe storingen in één IT-systeem doorwerken naar andere systemen binnen de procesketen?
    \end{enumerate}

    Het doel van dit onderzoek is om een end-to-end monitoringsoplossing te ontwerpen en te implementeren in de vorm van een proof-of-concept, die operationele IT-teams ondersteunt bij het tijdig detecteren van storingen en het sneller achterhalen van hun oorzaak. Door inzicht te bieden in kritieke IT-componenten en gebruik te maken van automatische alerting, beoogt de oplossing de operationele stabiliteit te verhogen en de impact van IT-storingen te beperken.


    
    \section{Literatuurstudie}%
    \label{sec:literatuurstudie}
    
    \subsection{Monitoring in moderne havenbedrijven}
    De afgelopen jaren hebben havens een sterke digitalisering doorgemaakt: processen die vroeger handmatig verliepen zijn nu grotendeels ondersteund door systemen die kunnen plannen, coördineren en beslissingen nemen. Havenbedrijven maken steeds meer gebruik van digitale technologieën om hun logistieke processen efficiënter en voorspelbaarder te organiseren. Door automatisatie worden processen sneller en beter op elkaar afgestemd, wat flexibiliteit verhoogt. Monitoring speelt hierin een belangrijke rol, omdat het organisaties toelaat om inzicht te krijgen in de werking van hun processen en IT-systemen. Tegelijk zorgt deze sterke afhankelijkheid van IT ervoor dat storingen een directe impact kunnen hebben op de operationele werking, wat nieuwe risico’s met zich meebrengt~\autocite{Rodrigue2024}.
    
    \subsection{Traditionele monitoringsmethoden en hun beperkingen}
    In veel IT-omgevingen wordt nog steeds gewerkt met “traditionele monitoringmethoden”, elk systeem wordt afzonderlijk gecontroleerd. Dit wordt vaak omschreven als component-based monitoring of infrastructure monitoring, gericht op individuele elementen zoals servers, databases, netwerkapparatuur of specifieke applicaties. Hoewel deze aanpak problemen binnen één systeem kan detecteren, ontstaan er belangrijke tekortkomingen wanneer bedrijfsprocessen uit meerdere systemen bestaan.
    Een opmerkelijk nadeel van traditionele monitoring is dat monitoringtools los van elkaar opereren, zonder gedeelde contextuele inzichten. Door dit gebeuren ontstaat er een gefragmenteerd beeld van incidenten en ontbreekt inzicht in hoe problemen in het ene systeem grote gevolgen kan hebben op het andere systeem. Dit zorgt voor een “blind spot” omdat organisaties de alerts zien op individuele systemen maar niet begrijpen wat de impact op het hele systeem is of kan zijn~\autocite{Brooks2004}.
    Nog een probleem is dat klassieke monitoring geen root-cause analyse biedt. Dat betekent dat teams vaak wel symptomen detecteren, maar dit niet kunnen herleiden naar waarom of hoe het incident ontstaan is~\autocite{Thalheim2017}. De schaalbaarheid speelt ook een rol in de beperkingen. Onderzoek toont aan dat veel bestaande monitoringsoplossingen, zowel commerciële als open-source, geen goede schaalbaarheid bieden. Dit heeft als gevolg dat het steeds moeilijker wordt om een representatief overzicht te behouden van het hele systeem. Dit kan ervoor zorgen dat de monitoringsoplossing minder goed werkt wanneer het systeem groter en complexer wordt. Oplossingen falen vaak in het bewaken van multi-tenant applicaties of componenten, wat resulteert in beperkte zichtbaarheid en minder betrouwbare incidentdetectie~\autocite{Andreolini2014}. Hierdoor zijn traditionele per-process monitoringsoplossingen onvoldoende voor complexe IT-landschappen waarin meerdere systemen samenkomen.
    
    \subsection{End-to-end monitoring}
    End-to-end monitoring richt zich op het bewaken van het hele traject in plaats van afzonderlijke componenten. Waar traditionele monitoringsmethoden enkel inzicht geven in individuele systemen, gaat end-to-end monitoring in het geheel bekijken hoe al deze systemen samenwerken binnen een bedrijfsproces~\autocite{Mireles2024}. Deze benadering maakt het mogelijk om zowel de prestaties van afzonderlijke applicaties als de onderlinge afhankelijkheden te volgen. Hierdoor kunnen afwijkingen sneller en accurater gedetecteerd worden. Door componenten zoals metrics, logs en traces te combineren, kan er een breder zicht op het volledig proces (A-Z) gecreëerd worden. Organisaties kunnen op deze manier sneller problemen signaleren, begrijpen en oplossen.
    Complexere systemen bestaan vaak uit vele onderling afhankelijke services, end-to-end observability wordt vooral gerealiseerd door “distributed tracing”. Individuele requests worden doorheen het volledige systeem gevolgd in combinatie met real-time logging, dit zorgt voor extra context en detail aan elke stap van het proces. Als je deze datasets gaat samenhangen ontstaat er een volledig beeld van hoe transacties door het hele proces lopen, wat er voor vertraging zorgt en waar het eventueel fout loopt. Dit is essentieel voor zowel nauwkeurige root-cause analyse als voor het verhogen van de betrouwbaarheid en stabiliteit van complexe IT-processen~\autocite{Saminathan2021}.
    
    \subsection{Operationele impact & KPI’s}
    Om de effectiviteit van een end-to-end monitoringsoplossing te evalueren zijn er duidelijke KPI’s.
    Een belangrijke KPI is de Mean Time to Detect (MTTD), die geeft aan hoe snel een incident wordt opgemerkt. Een lagere MTTD verkleint de kans op verstoringen en voorkomt escalatie. Daarnaast is er Mean Time to Repair (MTTR). Dit meet hoe snel een incident na detectie wordt hersteld, dit draagt direct bij aan de continuïteit van kritieke processen~\autocite{Chaplin2025}. Ook is beschikbaarheid een cruciale indicator, een hoge beschikbaarheid verhoogt de betrouwbaarheid van services en minimaliseert operationele\\
    risico’s~\autocite{Raza2025}.
    
    \subsection{Conclusie}
    Uit de literatuur blijkt dat moderne havenbedrijven sterk afhankelijk zijn van een gedigitaliseerd verbonden IT-landschap waarin systemen zoals Warehouse Management Systems (WMS) en Transport Management Systems (TMS) gezamenlijk instaan voor de operationele aansturing van magazijn- en transportprocessen. Digitalisatie verhoogt de efficiëntie en voorspelbaarheid maar het brengt ook risico’s met zich mee. Een verstoring bij één onderdeel kan een keteneffect veroorzaken dat de operatie vertraagt. Traditionele monitoring schiet hierin tekort omdat het geen overzicht heeft over het hele plaatje, onder meer door beperkte schaalbaarheid en onvoldoende inzicht in onderlinge afhankelijkheden. End-to-end monitoring zorgt bij dit probleem voor een oplossing, het bewaakt het volledige plaatje en kan sneller afwijkingen detecteren. KPI’s zoals MTTD, MTTR en beschikbaarheid worden hierdoor beter beheerst.
    
    
    \section{Methodologie}%
    \label{sec:methodologie}
    
    \FloatBarrier
    \begin{figure}[h!]
        \centering
        \includegraphics[width=\linewidth, keepaspectratio]{gantt.png}
        \caption{Gantt diagram met de verschillende fasen van het onderzoek.}
        \label{fig:gantt}
    \end{figure}
    
    Deze methodologie beschrijft de stappen die worden gemaakt om te onderzoeken in welke mate een end-to-end monitoringsoplossing kan bijdragen aan snellere incidentdetectie en hogere stabiliteit binnen het IT-landschap in de haven. 
    \subsection{Literatuurstudie}
    In de eerste fase wordt een literatuurstudie uitgevoerd naar bestaande monitoringsstrategieën, met focus op de tekortkomingen van traditionele monitoring en de voordelen van end-to-end monitoring. Er wordt onderzocht welke KPI’s algemeen gebruikt worden om monitoringsprestaties te evalueren (MTTD, MTTR en availability) en hoe deze kunnen worden toegepast in complexe, onderling afhankelijke IT-omgevingen zoals in havensystemen.
    
    \textbf{Deliverable:} Synthese van de theoretische basis, inclusief definitie van KPI’s en argumentatie voor end-to-end monitoring als oplossing.
    \textbf{Tijdsschatting:} 6 weken, 1 dag/week
    
    
    \subsection{Analyse van het bestaande IT-land\-schap}
    In deze fase wordt de huidige monitoringspraktijk binnen het bedrijf geanalyseerd. Dit omvat:
    \begin{itemize}
        \item Opnemen van kritieke systemen (WMS, TMS, infrastructuurcomponenten)
        \item Identificatie van blind spots, vertragingen in incidentdetectie
        \item Documenteren van incidenten om huidige MTTR-waarden te reconstrueren
    \end{itemize}
    Deze resultaten worden gebruikt om de functionele eisen van een end-to-end oplossing te bepalen.
    
    \textbf{Deliverable:} Overzicht van het bestaande monitoringslandschap en een lijst van functionele en niet-functionele vereisten voor de nieuwe oplossing.
    \textbf{Tijdsinschatting:} 4 weken, 1 dag/week
    
    \subsection{Proof-of-concept}
    Op basis van de analyse wordt een kleinschalige end-to-end monitoringsoplossing opgezet\\ waarin metrics, logs en eventueel traces uit de drie kritieke IT-componenten worden verzameld. Deze data wordt geïntegreerd in één dashboard, het dashboard wordt ook aangevuld met een alertingmechanisme dat binnen vijf minuten afwijkingen rapporteert. Door gesimuleerde incidenten uit te voeren, wordt er getest hoe snel en nauwkeurig de incidenten worden gedetecteerd en opgelost.
    
    \textbf{Deliverable:} Werkende proof-of-concept met dashboard en alertingsysteem, inclusief testresultaten.
    \textbf{Tijdsschatting:} 6 weken, 2 dagen/week.
    
    \subsection{Evaluatie}
    De proof-of-concept wordt geëvalueerd aan de hand van vooraf gedefinieerde KPI’s, waaronder MTTD, MTTR en availability. Tijdens gesimuleerde incidenten wordt gemeten hoe snel de incidenten gedetecteerd worden en hoe correct de gegenereerde alerts zijn. Deze resultaten worden vergeleken met de huidige monitoringsmanier om te bepalen of de end-to-end aanpak een meetbare verbetering oplevert. Binnen deze proof-of-concept wordt MTTD gemeten voor en na de implementatie van de monitoringoplossing. MTTR en availability worden niet met cijfers gemeten, maar besproken op basis van de testresultaten en het verkregen inzicht.
    
    \textbf{Deliverable:} Evaluatieverslag met vergelijking tussen huidige monitoring en de PoC.
    \textbf{Tijdsschatting:} 3 weken, 1 dag/week.
    
    \subsection{Synthese en conclusies}
    In de laatste fase worden de bevindingen uit de literatuurstudie, analyse, proof-of-concept en evaluatie samengebracht om een antwoord te formuleren op de onderzoeksvraag. Daarbij wordt er beoordeeld in welke mate end-to-end monitoring leidt tot snellere incidentdetectie en hogere operationele stabiliteit en de beperkingen van het onderzoek worden besproken en geformuleerd voor verdere implementatie.
    
    \textbf{Deliverable:} Eindrapport met conclusies, antwoorden op de onderzoeksvraag en eventuele aanbeveling voor verdere uitrolling.
    \textbf{Tijdsschatting:} 2 weken, 1 dag/week.
    
    \section{Verwachte resultaten}%
    \label{sec:verwachte-resultaten}
    
    Op basis van de literatuurstudie en de opzet van de proof-of-concept wordt er verwacht dat de end-to-end monitoringoplossing leidt tot een merkbare verbetering in incidentdetectie en operationeel inzicht. Concreet wordt een lagere MTTD verwacht omdat afwijkingen sneller opgemerkt worden door bepaalde metrics, logs en traces. Ook wordt een daling in MTTR verwacht, omdat we sneller de échte oorzaak vinden als we de keten beter begrijpen. Er wordt verwacht dat het dashboard een duidelijker overzicht van het volledige proces biedt, waardoor blind spots worden verminderd en de kans op incidenten wordt verkleind. Verder wordt aangenomen dat de proof-of-concept een hogere betrouwbaarheid van alerts aantoont met relevantere meldingen dan met de traditionele monitoringswijze. Tot slot wordt verwacht dat de resultaten inzicht zullen geven in welke mate end-to-end monitoring schaalbaar is naarmate het IT-landschap groter en complexer wordt, en toepasbaar is binnen de havencontext.
    
    \section{Conclusie}%
    \label{sec:conclusie}
    
    Dit onderzoek heeft als doel te bepalen in welke mate end-to-end monitoring kan bijdragen aan snellere incidentdetectie en -oplossing. Uit de literatuurstudie blijkt dat traditionele monitoring onvoldoende inzicht biedt in samenhangende processen, wat vaak leidt tot vertraagde detectie en beperkte root-cause analyse. End-to-end monitoring wordt in de literatuur naar voren geschoven als een veelbelovende aanpak om deze tekortkomingen aan te pakken, doordat het een geïntegreerd overzicht van het volledige systeem biedt.
    De proof-of-concept maakt het mogelijk om deze theoretische voordelen in de praktijk uit te proberen. De verwachte resultaten worden gebruikt om zo te kijken naar een potentiële oplossing voor lagere MTTD, efficiëntere analyse en een verbeterd overzicht van het volledige IT-landschap. Het onderzoek richt zich op het beoordelen van de haalbaarheid van end-to-end monitoring binnen complexere organisaties.
    
    
    \printbibliography[heading=bibintoc]
    
\end{document}
