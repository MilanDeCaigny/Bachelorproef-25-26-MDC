%==============================================================================
% Voorbeeld hogent-article: onderzoeksvoorstel bachproef
%==============================================================================

\documentclass{hogent-article}

% Invoegen bibliografiebestand
\addbibresource{voorstel.bib}

% Informatie over de opleiding, het vak en soort opdracht
\studyprogramme{Professionele bachelor toegepaste informatica}
\course{Bachelorproef}
\assignmenttype{Onderzoeksvoorstel}
\academicyear{2025-2026} % TODO: pas het academiejaar aan

% TODO: Werktitel
\title{Hoe kan een end-to-end monitoringoplossing worden ontworpen en geïmplementeerd om IT-incidenten tijdig te detecteren en sneller op te lossen binnen het havenlandschap? }

% TODO: Studentnaam en emailadres invullen
\author{Milan De Caigny}
\email{milan.decaigny@student.hogent.be}

% TODO: Medestudent
% Gaat het om een bachelorproef in samenwerking met een student in een andere
% opleiding? Geef dan de naam en emailadres hier
% \author{Yasmine Alaoui}
% \email{yasmine.alaoui@student.hogent.be}

% TODO: Geef de co-promotor op
\projectrepo{https://github.com/MilanDeCaigny/Bachelorproef-25-26-MDC}
\supervisor[Co-promotor]{F. Baert (Sea-invest, \href{mailto:frederik.baert@sea-invest.com}{frederik.baert@sea-invest.com})}

% Binnen welke specialisatierichting uit 3TI situeert dit onderzoek zich?
% Kies uit deze lijst:
%
% - Mobile \& Enterprise development
% - AI \& Data Engineering
% - Functional \& Business Analysis
% - System \& Network Administrator
% - Mainframe Expert
% - Als het onderzoek niet past binnen een van deze domeinen specifieer je deze
%   zelf
%
\specialisation{System \& Network Administrator}
\keywords{End-to-end monitoring, IT-landschap, Incidentdetectie, havenbedrijven }

\begin{document}
    
    \begin{abstract}
        Havenbedrijven vertrouwen op een sterk gedigitaliseerd IT-landschap waarin systemen zoals Warehouse Management Systems (WMS), die magazijnprocessen ondersteunen, en Transport Management Systems (TMS), die transportprocessen beheren en optimaliseren, een centrale rol spelen in de operationele aansturing. Storingen binnen één onderdeel kunnen hierdoor snel problemen bij andere onderdelen veroorzaken.
        Traditionele monitoringsmethoden bieden hierbij onvoldoende inzicht en leiden vaak tot vertraagde incidentdetectie. Dit onderzoek richt zich op de vraag in welke mate een end-to-end monitoringsoplossing kan bijdragen aan snellere incidentdetectie en hogere stabiliteit bij het IT-landschap in de haven. Het onderzoek omvat een literatuurstudie, analyse en proof-of-concept met metrics, logs en traces. Er wordt verwacht dat end-to-end monitoring leidt tot een lagere Mean Time to Detect (MTTD), zijnde de gemiddelde tijd die verstrijkt tussen het optreden van een IT-incident en het moment waarop dit incident door het monitoringssysteem wordt gedetecteerd. Daarnaast wordt een verbeterde root-cause analyse en een duidelijker ketenoverzicht verwacht, waarbij inzicht ontstaat in de onderlinge afhankelijkheden en interacties tussen de betrokken IT-systemen.
    \end{abstract}
